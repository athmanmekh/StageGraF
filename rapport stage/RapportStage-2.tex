\documentclass[12pt]{article}
\usepackage{geometry}
\geometry{hmargin=2cm,vmargin=1.5cm}

\usepackage{hyperref}
\usepackage[T1]{fontenc}
\usepackage[utf8]{inputenc}
\usepackage[francais]{babel}

\usepackage{graphicx}

\title{Rapport De Stage} %doesn't appear on PDF
\title{Licence Informatique 3\up{ème} année}
\author{MEKHZOUMI Athman} %doesn't appear on PDF
\author{Dirigé par Olivier Baudon} 

\begin{document}
\newpage
\maketitle
\tableofcontents
\newpage

\section{Introduction}

Dans cette partie, nous allons présenter le thème du stage, les prérequis exigés, le travaille demandé et dans quel contexte ce sujet a-t-il été posé.

\newline %don't work
\subsection{Sujet du stage}

Le sujet du stage est \textit{Publier une bibliothèque sur les graphes et la compléter}, une version beta de la bibliothèque était déjà implémenté avant le début du stage, les taches principales était de compléter le code là ou il y manquait, améliorer le code en s'inspirant des nouvelles version du langage \textit{java} et aussi documenter ansi que commenter la bibliothèque.
\newline Des connaissance en "Programmation orienté objet" et en "Théorie des graphes" m'étais nécessaire pour comprendre la conception et l'implémentation de la bibliothèque, heureusement pour j'avais déjà aqcuis les notions de base sur ces deux domaines au cours de mes études universitaires dans mon pays natale, mais cela ne m'été pas suffisant car il me fallait des connaissances plus profondes, alors j'ai lu quelques cours en ligne et codé programmes avant le début du stage pour étre au niveau éxigé. 

\newline %don't work
\subsection{Le contexte du stage}

Ce stage rentre dans le cadre du cycle licence, c'est stage obligatoire de 4 semaines minimum, il resprésente l'élément pédagogique 4TTVP18U et il fait partie du semèstre 6 du parcours informatique, l'obtention de ce stage été faite avec une candidature spontané adressé au Maître de Conférences Mr BAUDON Olivier qui fait partie de l'équipe des chercheurs du LaBRI.

\newpage

\section{Déroulement du stage}

Dans cette partie nous allons parler du travaille fait par l'étudiant, des compétences qui ont été mobilisé pour le réaliser ansi que les connaissances et compétences acquis et métrisé .

\subsection{Premiere rendu}



\subsection{Deuxième rendu}
\subsection{Troisième rendu}
\subsection{Quatrième rendu}

\section{Bilan du stage}

\end{document}
