\documentclass[12pt]{article}
\usepackage{geometry}
\geometry{hmargin=2cm,vmargin=1.5cm}

\usepackage{hyperref}
\usepackage[T1]{fontenc}
\usepackage[utf8]{inputenc}
\usepackage[francais]{babel}

\usepackage{graphicx}

\title{Rapport De Stage}
\title{Licence Informatique 3\up{ème} année}
\author{MEKHZOUMI Athman}
\author{Dirigé par Olivier Baudon}

\begin{document}
\maketitle
\tableofcontents
\newpage

\section{Introduction}

Dans cette partie, nous allons présenter le thème du stage et dans quel contexte ce sujet a-t-il été posé.

\subsection{Sujet du stage}

Le sujet du stage est \textit{\textbf{Publier une bibliothèque sur les graphes et la compléter}}, une version beta de la bibliothèque était déjà implémenté avant le début du stage, donc les taches principales était de la compléter le code là ou il manquait, améliorer le code en s'inspirant des nouvelles version du langage \textit{java} et aussi documenter ansi que commenter le code.

\subsection{Le contexte du stage}

\section{Déroulement du stage}
\subsection{Premiere rendu}
\subsection{Deuxième rendu}
\subsection{Troisième rendu}
\subsection{Quatrième rendu}

\section{Bilan du stage}

\end{document}
